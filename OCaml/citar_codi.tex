\usepackage{listings}
\usepackage{xcolor}

% Definició de colors
\definecolor{SkyBlue}{RGB}{0,122,204}     % Blau clar per a paraules clau
\definecolor{Orange}{RGB}{255,140,0}      % Taronja suau per a cadenes
\definecolor{SeaGreen}{RGB}{46,139,87}    % Verd apagat per a comentaris
\definecolor{LightGray}{RGB}{245,245,245} % Gris molt clar per al fons
\definecolor{MediumGray}{RGB}{160,160,160} % Gris mitjà per als números de línia
\definecolor{DarkPurple}{RGB}{128,0,128}  % Violeta fosc per a funcions
\definecolor{SoftYellow}{RGB}{255,255,150} % Groc suau per a literals

% Configuració de listings
\lstset{
    basicstyle=\ttfamily\footnotesize,      % Font bàsica
    keywordstyle=\color{SkyBlue},           % Paraules clau en Blau clar
    stringstyle=\color{Orange},             % Cadenes en Taronja suau
    commentstyle=\color{SeaGreen}\itshape,  % Comentaris en Verd apagat
    numbers=left,                           % Números de línia a l'esquerra
    numberstyle=\tiny\color{MediumGray},    % Estil de números de línia en gris mitjà
    stepnumber=1,                           % Cada línia numerada
    backgroundcolor=\color{LightGray},      % Fons gris molt clar
    breaklines=true,                        % Trencar línies llargues
    showstringspaces=false,                 % No mostrar espais a les cadenes
    frame=single,                           % Marc simple al voltant del codi
    frameround=tttt,                        % Cantons arrodonits
    xleftmargin=2.5em,                      % Marge esquerre
    xrightmargin=2.5em,                     % Marge dret
    captionpos=b,                           % Posició de la llegenda (bottom)
    identifierstyle=\color{DarkPurple},     % Funcions en Violeta fosc
    moredelim=**[is][\color{SoftYellow}]{@}{@},  % Groc suau per a literals dins del codi
}


\begin{lstlisting}[language=C++,caption={Exemple d'un programa que calcula la suma d'un vector d'enters en C++.}, label={fig:C++}]
// Suma de tots els elements d'un vector d'enters
int main() {
    int total = 0;                        //total de la suma
    vector<int> vec {1, 2, 3, 4};         //vector d'enters
    for(int i = 0; i < vec.size(); ++i){  //per a tot element de vec
        total += vec[i];                  //incrementem el valor
    }
    cout << "Total = " << total << endl;  //imprimim el resultat
}                                         //Total = 10
\end{lstlisting}

\begin{lstlisting}[language=Haskell, caption={Exemple d'un programa que suma una llista d'enters en Haskell.}, label={fig:Haskell}]
--Suma de tots els elements d'una llista d'enters
sumList :: [Int] -> Int  --La funcio rep una llista i retorna un int
sumList [] = 0                      --Cas base: La llista esta buida
sumList (x:xs) = x + sumList xs     --Cas recursiu

main :: IO ()
main = print (sumList [1, 2, 3, 4]) --Imprimeix la suma dels 
                                    --elements (10)
\end{lstlisting}
